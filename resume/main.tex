\documentclass[10pt, a4paper]{cvhari}
\geometry{left=1.2cm,right=1.2cm,marginparwidth=6.8cm,marginparsep=1.2cm,top=1.2cm,bottom=1.2cm}
\usepackage[default]{lato}
\begin{document}

\name{Chang, Yunbin} 
\tagline{Vision \& Software Engineer} % 
\personalinfo{%
    \mailaddress{yvin.chang@gmail.com}
    \phone{+82 10-4107-3727}
    \location{Seoul, Republic of Korea}
    \linkedin{linkedin.com/in/vinchang}
}
\makecvheader
\medskip

\cvsection{Profile}
\textbf{백엔드 및 머신 비전 엔지니어}로서, 백엔드 개발, 비전 소프트웨어, 하드웨어 연동 시스템 분야에서 2년 이상의 실무 경험을 보유하고 있습니다. 어릴 적부터 컴퓨터 과학에 대한 열정을 바탕으로 소프트웨어 특기자 전형으로 입학하였으며, 이공계 국가우수장학생으로 조기 졸업하였습니다. 학부 재학 중에도 연구실과 기업에 재직하며 실무 프로젝트에 지속적으로 참여해 실전 감각을 쌓았고, 반도체 검사 및 이커머스 등 다양한 분야에서 여러 프로젝트를 처음부터 끝까지 주도적으로 수행하였습니다. 특히 \textit{Deepseers}에서 주도한 비전 검사 프로젝트는 중소벤처기업부의 \textit{TIPS 및 DIPS R\&D} 프로그램에 연속 선정되어 기술적 우수성과 시장성을 동시에 인정받았습니다. 메모리 효율적인 \textit{GAN} 프레임워크를 연구하고, 딥러닝 기반 \textit{ROI} 추천 기술로 특허를 출원하며 \textit{AI} 분야도 함께 탐구해 왔습니다. 더불어, 주한미군사령부에서의 카투사 복무를 통해 다국적 인사 데이터 관리 및 협업 경험을 쌓으며, 글로벌 환경에서도 능동적으로 업무를 수행할 수 있는 역량을 갖추었습니다.

\smallskip

\cvsection{Summary Of Experience}
\begin{itemize}
  \item \textbf{Scalable Architecture Design}
    \begin{itemize}
      \item 다양한 산업용 하드웨어에 유연하게 적용 가능한 모듈형 C\#/.NET 기반 비전 소프트웨어를 주도적으로 개발하여, 기존의 레거시 시스템을 대체하고 현장 셋업을 간소화하여 엔지니어링 부담을 크게 줄였습니다.
      \item 다양한 장비의 추가를 위한 모듈화된 확장성을 보장하였습니다.
      \item 동적 레시피를 활용하여 여러 자재 종류와 검사 시나리오에 유연하게 대응할 수 있는 구조를 설계하였습니다.
      \item SQLite을 활용하여 검사 히스토리 데이터베이스 및 LOT 추적 시스템을 개발하였습니다.
      \item 실시간 불량 맵핑 및 데이터 동기화를 위한 EMAP/X-out upstream을 연동하였습니다.
    \end{itemize}
  \item \textbf{Hardware-Integrated Software Development}
    \begin{itemize}
      \item 비전 소프트웨어와 장비 핸들러 간 UDP 통신을 구성하고, 프로토콜에 따라 컨트롤러와 연동되도록 구현하였습니다.
      \item Camera Link, GigE 등 산업용 카메라, 스트로브 조명, 트리거 기반의 조명 제어 및 이미지 취득 시스템을 구현하였습니다.
      \item 옵저버 기반 블로킹 큐를 활용한 비동기 트리거 처리 파이프라인을 설계하여 고속 검사 시 프레임 손실을 방지하였습니다.
      \item 스트로브 조명과 트리거를 이용해 고속 플라잉 검사 기능을 구현하여, SW 트리거 방식 대비 속도를 크게 향상시켰습니다.
      \item RS232 기반의 다양한 조명 컨트롤러를 연동하고, 레시피에 따라 자동으로 조명이 제어되도록 구현하였습니다.
    \end{itemize}
  \item \textbf{Machine Vision}
    \begin{itemize}
      \item C\#, WPF, Halcon을 활용하여 반도체 패키지 검사 알고리즘을 개발 및 최적화하였습니다.
      \item 결함 유형별 최적화된 다양한 조명 조건에서 이미지를 획득하는 멀티샷 조명 전략을 설계하여 단일 조명 대비 검출 성능을 향상시켰습니다.
      \item 고해상도 이미지 합성을 위한 부분 영역 패치 기반, 메모리 효율적인 조건부 GAN 프레임워크를 개발하였습니다.
    \end{itemize}
  \item \textbf{Backend \& Cloud Infrastructure Development}
    \begin{itemize}
      \item 드랍쉬핑 플랫폼의 물류, 결제, 관리자 시스템을 통합한 확장 가능한 백엔드 및 인프라를 풀스택으로 구축하였습니다.
      \item Node.js와 TypeScript를 활용하여 “THEKOC” 드랍쉬핑 플랫폼의 RESTful API를 설계 및 구현하였습니다.
      \item 고가용성 서비스를 지원하기 위해 AWS 인프라(EC2, S3, LB, RDS, ElastiCache, Nginx)를 설계하였습니다.
      \item 상품, 주문, 리뷰 관리 모듈을 개발하고, SCM 시스템을 통한 QExpress 창고와의 연동으로 배송 업무를 자동화하여 드랍쉬핑 전 과정을 효율화하였습니다.
    \end{itemize}
\end{itemize}

\cvsection{Education}
\smallskip
    \education
        {Hansung University,\\ B.S. in Computer Science and Engineering}
        {
            Honors: GPA: 4.27/4.5 (Major 4.43)
            \begin{itemize}
                \item 조기 졸업
                \item 소프트웨어 특기자 전형 입학
                \item 국가우수장학금(이공계)
                \begin{itemize}
                    \item 3, 4학년 전액 장학금 수혜
                \end{itemize}
                \item 최우수한성인재 장학금
                \begin{itemize}
                    \item 2학년 70\% 장학금 수혜
                \end{itemize}
                \item 우수한성역량장학금
                \begin{itemize}
                    \item 1학년 1학기 50\%, 2학기 30\% 장학금 수혜
                \end{itemize}
            \end{itemize}
        }
        {Seoul, South Korea}
        {Mar 2019 - Aug 2024}
        \par

\cvsection{Professional \& Other Experience}

    \company
        {Deepseers}
        {Vision \& Software Engineer}
        {Seoul, South Korea}
        {Feb 2024 -- Till now}
    \project
        {반도체 후공정 패키지 비전 검사 솔루션}
        {Genesem}
        {Software Engineer}
        {C\#, WPF, Halcon, SQLite}
    
    \begin{itemize}
        \item 대학 연구실에서 스핀오프된 스타트업에 합류하여 기존 프로젝트와 역할을 지속적으로 수행하였습니다.
    \end{itemize}
    \vspace{1mm}
    \begin{itemize}
        \item \textsl{시스템 아키텍처 및 핵심 모듈 개발}
        \begin{itemize}
            \item 시스템 전체 아키텍처를 처음부터 설계하고, 핵심 모듈 개발을 기술적으로 리드하였습니다.
            \item 고속 플라잉 검사 환경을 위해 Observer + Blocking Queue 기반 트리거-그랩 동기화 구조를 구현하였습니다.
            \item 다양한 자재에 적응 가능한 레시피 구조를 개발하고, EMAP/X-out 등 상위 데이터 흐름을 통합하였습니다.
            \item LOT 관리 및 검사 이력 추적을 위한 데이터베이스를 구축하였습니다.
        \end{itemize}
    \end{itemize}
    \begin{itemize}
        \item \textsl{비전 검사 및 알고리즘 개발}
        \begin{itemize}
            \item 맵핑 비전 검사의 모든 검사 알고리즘(마킹면 포함)개발을 아키텍처 설계부터 배포까지 주도하였습니다.
            \item 맵핑 검사와 PRS 과정에서 바텀면과 마킹면 검사가 자유롭게 전환되도록 구현하여, 장비 운용의 유연성을 향상시켰습니다.
            \item 유효한 핵심 영역만을 학습 대상으로 활용하여 Shape 매칭 모델의 연산 효율을 높이고 얼라인 시간을 단축하는 등, 다양한 비전 검사 알고리즘의 최적화에 기여하였습니다.
            \item 딥러닝 기반 ROI 추천 알고리즘에 대해 특허를 출원하였습니다.
        \end{itemize}
    \end{itemize}
    \begin{itemize}
        \item \textsl{하드웨어 연동 및 제어}
        \begin{itemize}
            \item 트리거/스트로브 조명 제어를 구현하고, 하드웨어 트러블슈팅을 수행하였습니다.
            \item 동적 설정 구조를 통해 다양한 카메라 인터페이스와 조명을 유연하게 통합하여, 장비 셋업 부담을 최소화하였습니다.
            \item 장비 타입별 핸들러-비전 소켓 및 라우팅 로직을 개발하여 다양한 장비 타입(Tray, Ring 등)과의 호환성을 확보하였습니다.
            \item 핸들러와 비전 시스템 간 LOT 및 레시피 동기화, 이력 관리를 개발하였습니다.
        \end{itemize}
    \end{itemize} 
        
\smallskip
\divider
\smallskip
%\clearpage

    \company
        {AML Lab. Hansung University}
        {Undergraduate Researcher}
        {Seoul, South Korea}
        {May 2023 - Feb 2024}
        \project
            {반도체 후공정 패키지 비전 검사 솔루션}
            {N/A}
            {Software Engineer}
            {C\#, WPF, Halcon, SQLite}
        \begin{itemize}
            \item 반도체 패키지 장비(Saw Singulation)용 비전 검사 소프트웨어를 초기 프로토타입부터 배포까지 개발하였습니다.
            \item 실제 반도체 장비에 프로토타입 소프트웨어를 배포 및 검증하여 안정적이고 실시간으로 동작함을 확인하였습니다.
            \item 반도체 패키지 맵핑 검사에서 수행되는 모든 알고리즘(마킹면 포함) 개발에 기여하였습니다.
            \item 협업 테스트와 피드백을 통해 바닥면 검사 알고리즘의 구현 및 파라미터 최적화 과정에 기여하였습니다.
        \end{itemize} 

        
        \project
            {메모리 효율적인 GAN 프레임워크 연구}
            {N/A}
            {Researcher}
            {Python, Pytorch, Numpy}
        
        \begin{itemize}
            \item 고해상도 이미지 생성을 위한 패치 기반 메모리 효율적 GAN 프레임워크를 연구하였습니다.
            \item KCI 등재 학술지에 논문 1편을 게재하였습니다.
            \item 국내(구두) 1건, 국제(포스터) 1건의 학술대회에서 발표하였습니다.
            \item HCI2024 구두 발표에서 최우수 논문상을 수상하였습니다.
        \end{itemize} 


\smallskip
\divider
\smallskip

    \company
        {STAFACT INC.}
        {Backend Developer}
        {Seoul, South Korea}
        {Jun 2022 - May 2023}
        \project
        {Dropshipping Platform "THEKOC"}
        {In-house}
        {Backend Developer}
        {Node.Js, TypeScript, MySQL, AWS}
        
        \begin{itemize}
            \item 모든 public/private API 개발을 단독으로 담당하였습니다.
            \item AWS EC2, S3, LB, RDS, ElastiCache, Nginx 등으로 구성된 AWS 기반 클라우드 인프라를 구축 및 운영하여 고가용성을 보장하였습니다.
            \item 카카오, 네이버, 구글, 애플, 페이스북 소셜 로그인 연동을 구현하였습니다.
            \item PayPal, WeChat Pay, MoMo 등 글로벌 결제 시스템을 연동하였습니다.
            \item 토스페이먼츠, 카카오페이, 네이버페이 등 국내 결제 시스템을 연동하였습니다.
            \item 관리자 대시보드 및 관리 기능을 전면 개발하였습니다.
            \item 판매자 미니샵 생성 및 맞춤형 가격 책정 기능을 구현하였습니다.
            \item 카카오톡 알림 및 인증 시스템을 개발하였습니다.
            \item 판매자 정산 환불 시스템을 구축하였습니다.
            \item Nuxt를 활용하여 프론트엔드 개발 커밋 기준 30\% 이상을 기여하였습니다.
        \end{itemize}

        \project
        {사내 SCM 시스템 개발}
        {In-house}
        {Backend Developer, QA Engineer}
        {Node.Js, TypeScript, MySQL, AWS}

        \begin{itemize}
            \item QExpress 창고와 SCM 간 자동 배송 연동을 구현하였습니다.
            \item 사내 SCM을 상품, 주문, 리뷰 관리 모듈과 연동하였습니다.
            \item 상품/주문 관리 및 인플루언서 체험단(샘플링) 기능을 개발하였습니다.
        \end{itemize}

\smallskip
\divider
\smallskip

    \company
        {United Stated Forces Korea(USFK), HQ\\ United Nations Command(UNC)}
        {KATUSA, J1 Data Management}
        {Camp Humphreys, South Korea}
        {Dec 2020 - Jun 2022}
        
        \projectwo
        {Database Management \& Personnel}
        {KATUSA, HR Specialist(42A)}
        
        \begin{itemize}
            \item PIMS-K(인사정보관리시스템)를 통해 10,000명 이상의 인사 기록을 관리하며, 여러 사령부에 걸쳐 데이터의 정확성과 무결성을 보장하였습니다.
            \item 미군 규정(USFK Regulation 60-1)에 따라 duty-free purchase violation tracking 시스템을 개선하고, USFK 전 지역 60,000명 이상의 수혜자를 대상으로 월간 보고 체계를 구축하였습니다.
            \item 미군 고위 리더십을 위해 Weekly Strength Report 작성과 번역을 지원하며, 한영 이중언어 업무를 수행하였습니다.
            \item 연합지휘소훈련(CCPT) 기간 동안 유엔군(UNC) 등 다국적 인원의 In-processing 및 인사 행정 절차를 관리하였습니다.
        \end{itemize} 

        \projectwo
        {Policy \& Program}
        {KATUSA, HR Specialist(42A)}
        
        \begin{itemize}
            \item 미 대사관 직원 및 군인 가족을 위한 규정 예외(ETP)를 200건 이상을 처리하였습니다.
            \item 미군과 연합사 간 인사 협조 및 공동 정책 정렬을 통해 협업을 강화하였습니다.
            \item 미군 및 유엔군을 대상으로 한 CFC, C5 주관 Cultural Immersion 프로그램 확대에 기여하여, 문화 적응력 및 양국 교류를 증진하였습니다.
        \end{itemize} 

        \bigskip
        \smallskip
        \hfill {\scriptsize \textit{(Note: Certain achievements are summarized in accordance with military security policy, as recognized in JSCM citation.)}}
    

    
\medskip

\cvsection{Awards}
    \bigskip
    
    \cvachievement{\faTrophy}{ {\bfseries Best Paper Award (Oral)} - HCI Korea 2024 \hfill { \color{accent} \faCalendar}  \hspace{0.5em}{Jan 2024}\smallskip}
    {HCI Korea 2024 학회에서 발표한 논문 "A Sub-Region Approach with Conditional GANs for Memory Efficiency Enhancement"가 최우수 논문상(Best Paper Award)으로 선정되었습니다.}\par
    \smallskip 
    \divider
    \smallskip
    
    \cvachievement{\faTrophy}{ {\bfseries Joint Service Commendation Medal} - 미국방부장관, United States Of America \hfill { \color{accent} \faCalendar}  \hspace{0.5em}{Apr 2022}\smallskip}
    {
        \begin{itemize}
            \item 전역 시 주한 미군/UN군 사령관 \textit{Paul J. LaCamera} 대장으로 부터 수여받음.
        \end{itemize}
        합동공로훈장(JSCM)은 미국 국방장관 명의로 수여되는 훈장으로, 육군, 해군, 공군, 해병대 등 여러 군이 협력하는 참모조직이나 연합사령부 등 합동 조직(joint staff)에서 근무하며 감독직(supervisory role) 또는 전략적 핵심 보직(key positions)을 맡아, 탁월한 업적이나 공로를 세운 군인에게 수여됩니다. 이 훈장은 각 군에서 개별적으로 수여하는 공로훈장보다 상위의 서열을 가지며, 이 훈장은 합동 참모 조직 등에서의 고위직 근무자를 대상으로 하며, 각 군별 공로훈장보다 높은 서열을 가집니다.




    }\par

    \smallskip 
    \divider
    \smallskip

    \cvachievement{\faTrophy}{ {\bfseries 국방부장관상: 국방부} - 2021 국방 공공데이터 활용 경진대회, 서비스 개발 부문 최우수 \hfill { \color{accent} \faCalendar}  \hspace{0.5em}{Aug 2021}\smallskip}
    {
        \begin{itemize}
            \item Inference API, 웹 RESTful API, Frontend 페이지를 개발하였습니다.
            \item tf-idf와 word2vec을 활용한 딥러닝 기반 개인화 추천 모델 개발을 지원하였습니다.
        \end{itemize}
        인공지능을 활용하여 군 장병의 체계적인 자기계발을 지원하는 플랫폼을 개발하였습니다. 플랫폼은 개인별 맞춤 도서 및 자격증을 추천하고, 자격증 합격률 정보를 제공합니다. 사용자는 마크다운(Markdown) 문법을 활용해 자기계발 일지를 작성하며, 자신의 성장 과정을 쉽게 관리할 수 있습니다. 또한, 포인트를 부여하는 랭킹 시스템을 도입하여 사용자 간의 건강한 경쟁을 유도하였습니다.
    }\par

    \smallskip 
    \divider
    \smallskip

    \cvachievement{\faTrophy}{ {\bfseries 최우수상: Mantech} - 제 2 회 오픈 인프라 개발 경진대회 (OIDC 2020) \hfill { \color{accent} \faCalendar}  \hspace{0.5em}{Aug 2020}\smallskip}
    {
        \begin{itemize}
            \item API 및 예측 모델을 포함한 모든 기능을 개발하였습니다.
        \end{itemize}
        서울시 공공자전거 ‘따릉이’의 실시간 대여 가능 정보를 정기적으로 수집하고 미래의 재고 변화를 예측하는 플랫폼을 개발하였습니다. 이 플랫폼은 서울시 Open API를 활용하여 실시간 데이터를 자동으로 수집하며, LSTM 딥러닝 모델을 통해 향후 대여소별 자전거 재고를 예측합니다. 이를 통해 사용자들이 자전거 이용 가능 여부를 사전에 파악할 수 있도록 하여, 따릉이 이용 편의성 및 접근성을 효과적으로 향상시켰습니다.

    }\par

    \smallskip 
    \divider
    \smallskip

    \cvachievement{\faTrophy}{ {\bfseries 특별상: 과학기술정보통신부, 국회도서관} - 제 2 회 개방형 클라우드 플랫폼 서비스 개발 해커톤 \hfill { \color{accent} \faCalendar}  \hspace{0.5em}{Dec 2019}\smallskip}
    {
        \begin{itemize}
            \item API 및 예측 모델을 포함한 모든 기능을 개발하였습니다.
        \end{itemize}
        실종 동물의 실시간 정보를 공유할 수 있는 위치 기반 플랫폼을 개발하였습니다. 사용자는 실종 동물의 현재 상태를 확인할 수 있으며, 마지막으로 목격된 위치나 최근 발견된 위치를 공유할 수 있습니다.
    }
    \medskip

\medskip

\cvsection{Patents}
\begin{itemize}
  \patententry
    {머신비전을 위한 딥러닝 기반 관심영역 자동생성 시스템 및 방법\\ \small (System and method for automatic creating region of interest based on deep learning for machine vision)\smallskip}
    {10-2024-0153587 - 우선심사 진행중}
    {K. Han, Y. Chang, K. Lee}
    {Nov 2024}
\end{itemize}

\bigskip

\cvsection{Publications and Presentations}
% — Journal Articles —
\textbf{Journal}\\
\begin{itemize}
  \journalentry
    {MAGICal Synthesis: Memory-Efficient Approach for Generative Semiconductor Package Image Construction}
    {Yunbin Chang, Wonyong Choi, Keejun Han*}
    {Journal of the Microelectronics and Packaging Society}
    {2023}
    \par
    \smallskip 
    \dividergray
    \smallskip
\end{itemize}


    
% — Conference Papers —
\textbf{Conferences}\\
\begin{itemize}
  \conferenceentry
    {A Sub-Region Approach with Conditional GANs for Memory Efficiency Enhancement}
    {Yunbin Chang, Wonyong Choi, Keejun Han*}
    {HCI 2024}
    {2024}

    \par
    \smallskip 
    \dividergray
    \smallskip

  \conferenceentry
    {A Computationally Optimized Data Augmentation Framework Utilizing cDCGAN for High-Resolution Package Images Acquisition}
    {Yunbin Chang, Wonyong Choi, Keejun Han*}
    {21st International Symposium on Microelectronics and Packaging (ISMP 2023)}
    {2023}
\end{itemize}

\bigskip

\cvsection{Research Projects}
  \researchentry
    {Jan~2025 -- Present}
    {Researcher, Deepseers}
    {사용자 개입이 최소화된 차세대 반도체 패키지 불량 검출 시스템}
    {(Next-generation semiconductor package defect detection system with minimal user intervention)}
    {중소벤처기업부(TIPS)}

  \par
  \smallskip 
  \dividergray
  \smallskip
  
  \researchentry
    {Oct~2024 -- Dec2024}
    {Researcher, Deepseers}
    {사용자 개입이 최소화된 차세대 반도체 패키지 불량 검출 시스템}
    {(Next-generation semiconductor package defect detection system with minimal user intervention)}
    {중소벤처기업부(TIPS)}

  \par
  \smallskip 
  \dividergray
  \smallskip

  \researchentry
    {May~2023 -- Dec2024}
    {Researcher, AML Lab.\ Hansung University}
    {반도체 소자 생산을 위한 인공지능 기반 스마트 제조 공정 장비 및 관련 기술 개발}
    {(Development of AI based smart manufacturing process and equipment technology to strengthen the competitiveness of semiconductor materials parts and equipment)}
    {산업통상자원부}

\bigskip

\cvsection{Certifications}
\begin{itemize}
  \item Nov 2017 \textbf{Linux Master II} \\
        Korean Association for ICT Promotion
  \bigskip
  \item Feb 2019 \textbf{Driver's License Class 2} \\
        Commissioner of Seoul Metropolitan Police Agency
\end{itemize}

\bigskip

\cvsection{Languages}
\begin{itemize}
  \item Korean – Native
  \item English – Intermediate (TOEIC 845)
\end{itemize}


\end{document}



